\documentclass[a4paper]{article}

\usepackage[english]{babel}
\usepackage[utf8]{inputenc}
\usepackage{amsmath}
\usepackage{graphicx}
\usepackage[colorinlistoftodos]{todonotes}

\title{User Guide}

\author{Corentin Arnaud}

\date{\today}

\begin{document}
\maketitle

\begin{abstract}
Your abstract.
\end{abstract}

\section{setup File}
They are three setup file, both in xml. The following subsection will describe them.
\subsection{Arm setup}
We use two setup file for describe the arm. One for the structure, the other for the muscles.
\subsubsection{Structure setup}
The name of the structure setup file is "ArmParamsDM.xml" where D is the degree of freedom of the arm and M the number of muscles.
The root of the file is <Arm>, it is compose of <setup>, <DampingTerm>, <MomentMatrix> and <Bounds>
<setup> is compose of four elements <Length>, <Mass>, <Inertia>, and <DistanceCenterBarycenter>. Each of this elements have two parameters and one child by segment of the arm. The parameters is  unit and type.
Name of Segment elements doesn't matters but they have to contain  

\subsection{Global setup}




\end{document}